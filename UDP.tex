\chapter{UDP}

\section{Overview of the UDP Implementation}

\subsection{Common Base Class (\texttt{L4\_UDP} and \texttt{L4\_UDP\_impl})}
The base class \texttt{L4\_UDP\_impl} contains common functionalities for UDP. Here's an overview of its main methods:

\subsubsection{\texttt{\large pr\_init()}}
\begin{itemize}
    \item Initializes the UDP protocol.
    \item Sets up necessary structures and initial states required for UDP operations.
    \item Allocates resources, initializes data structures, and configures default settings.
\end{itemize}

\subsubsection{\texttt{\large pr\_input()}}
\begin{itemize}
    \item Handles incoming UDP packets.
    \item Processes packets received from the network.
    \item Verifies checksums to ensure data integrity.
    \item Passes valid data to the socket layer.
\end{itemize}

\subsubsection{\texttt{\large pr\_output()}}
\begin{itemize}
    \item Prepares and sends outgoing UDP packets.
    \item Constructs the UDP header.
    \item Calculates the checksum to detect transmission errors.
    \item Sends the packet to the network.
\end{itemize}

\subsubsection{\texttt{\large pr\_usrreq()}}
\begin{itemize}
    \item Handles various user requests related to UDP operations.
    \item Manages opening and closing sockets, binding to ports, and sending data.
    \item Serves as the interface between user-level operations and the UDP implementation.
    \item Processes and fulfills user requests, ensuring correct socket configuration.
\end{itemize}

\subsubsection{\texttt{\large udp\_attach()}}
\begin{itemize}
    \item Attaches a UDP control block to a socket.
    \item Associates a socket with the necessary protocol control structures.
    \item Sets up the control block with initial parameters and links it to the socket.
\end{itemize}

\subsubsection{\texttt{\large udp\_output()}}
\begin{itemize}
    \item Manages the preparation and transmission of UDP packets.
    \item Constructs the packet and sets appropriate headers.
    \item Calculates the checksum and sends the packet to the network layer.
    \item Ensures correct encapsulation and transmission according to UDP standards.
\end{itemize}

\subsubsection{\texttt{\large udp\_template()}}
\begin{itemize}
    \item Creates a template for sending UDP packets on a connection.
    \item Allocates a buffer for a skeletal UDP/IP header.
    \item Minimizes the amount of work necessary for connection use.
\end{itemize}

\subsubsection{\texttt{\large insque()}}
\begin{itemize}
    \item Inserts an object into a queue.
    \item Manages linked lists of protocol control blocks.
    \item Ensures correct linking of the object to previous and next objects in the list.
\end{itemize}

\subsubsection{\texttt{\large remque()}}
\begin{itemize}
    \item Removes an object from a queue without deleting it.
    \item Updates pointers of previous and next objects to bypass the removed object.
    \item Maintains the integrity of the linked list.
\end{itemize}

\subsubsection{\texttt{\large in\_pcblookup()}}
\begin{itemize}
    \item Looks up protocol control blocks (PCBs) based on criteria like source and destination addresses and ports.
    \item Essential for locating existing connections.
    \item Ensures packets are delivered to the correct endpoint.
\end{itemize}

\section{Implementation Details of L4\_UDP\_Impl}

\subsection{UDP Header (\texttt{udphdr})}
\begin{itemize}
    \item Defines the header of a UDP packet.
    \item Includes source and destination ports, length, and checksum.
    \item Ensures necessary information for proper delivery and error checking.
\end{itemize}

\subsection{UDP Pseudo Header (\texttt{udpiphdr})}
\begin{itemize}
    \item Combines UDP and IP headers to provide a complete packet structure.
    \item Includes source and destination IP addresses, protocol information, and the UDP header.
\end{itemize}

\subsection{UDP Control Block (\texttt{udpcb})}
\begin{itemize}
    \item Manages control information for each UDP connection.
    \item Includes functions for initializing, attaching, and managing the state of UDP connections.
    \item Ensures proper maintenance and handling of incoming and outgoing packets.
\end{itemize}

\subsection{Logger (\texttt{udpcb\_logger})}
\begin{itemize}
    \item Handles logging of UDP control block activities.
    \item Records events and state changes for debugging and analysis.
    \item Aids in identifying issues and optimizing performance.
\end{itemize}
